% !TEX root = ./main.tex
\section{Actieplan} % (fold)
\label{sec:actieplan}
\subsection{Mirror als groeiende lean onderneming} % (fold)
\label{sub:mirror_als_groeiende_lean_onderneming}
Mirror volgt de principes van lean ondernemen\cite{lean-boekje}\cite{lean-enterprise}. De ontwikkeling van ons concept verloopt simultaan met de ontwikkeling ervan.  We houden onze investeringskost beperkt in de eerste fase van onze onderneming  tot we zeker zijn van de haalbaarheid van onze dienst/product, we passen ons aanbod aan op de behoefte van de klant en we ondernemen groeiend met voortdurende aandacht voor kwaliteitsverbetering.
% subsection mirror_als_groeiende_lean_onderneming (end)

\subsection{Minimum Viable Product: uittesten en feedback} % (fold)
\label{sub:minimum_viable_product_uittesten_en_feedback}
Mirror gaat in eerste fase op de markt met de pure essentie van onze dienst/product en lokt feedback uit. In deze fase ligt ons accent op de dienstverlening in eigen beheer. Voor de productie werken we samen met leveranciers (bestaande 3D-printing- en lasercuttingcentra in en buiten de regio).

De prioritaire targetgroepen waar we ons in deze fase op richten en waar we feedback bij uitlokken zijn targetgroep 1 en 2:

\begin{itemize}
  \item Makers erkend door `Handmade in Brugge'
  \item Niet erkende makers door `Handmade in Brugge' in dezelfde sectoren
\end{itemize}

In deze fase willen we ontdekken wat onze klanten van onze dienstverlening en ons product vinden, waar zij mogelijkheden en kansen zien, wat zij overbodig vinden …
% subsection minimum_viable_product_uittesten_en_feedback (end)

\subsection{Ontwikkelfases} % (fold)
\label{sub:ontwikkelfases}
\begin{enumerate}
  \item Bouwen en ontwikkelen: We bouwen aan onze dienstverlening/product door er nieuwe mogelijkheden aan toe te voegen. Deze vernieuwingen laten we los op onze klanten.
  \item Luisteren en meten: We gaan het gesprek aan met de klant en proberen resultaten zo objectief mogelijk te meten.
  \item Ontwikkelen: Met de verkregen feedback gaan we gericht aan de slag: we behouden aanpassingen of gooien deze overboord.
\end{enumerate}

Deze fases herhalen we meermaals. Zo valt het ontwikkelen van ons concept terzelfder samen met de commercialisering ervan.
% subsection ontwikkelfases (end)
\subsection{Startfase: pop-up} % (fold)
\label{sub:startfase_pop_up}
Mirror zet zich met een knal in de Brugse regio.

In deze fase ligt ons accent op de dienstverlening in eigen beheer. Voor de productie werken we samen met leveranciers.

De prioritaire targetgroepen waar we ons in deze fase op richten en waar we feedback bij uitlokken zijn targetgroep 1 en 2:

\begin{itemize}
  \item Makers erkend door `Handmade in Brugge'
  \item Niet erkende makers door `Handmade in Brugge' in dezelfde sectoren
\end{itemize}

In deze fase willen we ontdekken wat onze klanten van onze dienstverlening en ons product vinden, waar zij mogelijkheden en kansen zien, wat zij overbodig vinden …

Mirror zet zich opvallend in de markt en kiest hiervoor een pop-up locatie in een zone in de stad waar makers aan de slag zijn, waar vernieuwende ondernemers starten (bv. Ezelstraat). Het voordeel aan een pop-up is dat we zo de nieuwsgierigheid van makers kunnen prikkelen zonder de langdurige en hoge vaste kost van een winkelpand.

Mirror profileert zich niet als ontwerpbureau, maar zet zichzelf in de schaduw van ontwerpers en creatieve breinen.  De makers (erkend door `Handmade in Brugge' en ander) ervaren op dit moment misschien nog geen nood aan de diensten van Mirror. Door verschillende evenementen te organiseren i.s.m. `Handmade in Brugge' , de Stad Brugge creëren we een hype en verleiden we zo de makers uit de Brugse regio ons te ontdekken.

\subsubsection{Teasers voor makers erkend door `Handmade in Brugge'} % (fold)
\label{ssub:teasers_voor_makers_erkend_door_handmade_in_brugge}

Mirror ontwerpt voor de door `Handmade in Brugge' erkende makers een persoonlijke teaser. Deze gepersonaliseerde teasers tonen de kansen en mogelijkheden van Mirror voor de makers.

De teaser worden gefotografeerd en digitaal aan de bedrijven bezorgd. De aangeschreven makers worden persoonlijk uitgenodigd hun gepersonaliseerde teaser op afspraak af te halen in de pop-up die gedurende 3-6 maanden in een alternatieve straat van de stad is gevestigd. De periode van de pop-up leggen we in het begin van een kalenderjaar (een periode van plannen voor makers).
Bij het afhalen van deze persoonlijke  teaser wordt in een gesprek de mogelijkheden van Mirror toegelicht. Tevens wordt naar de mogelijke andere noden van de potentiële klanten gepeild.

Wanneer de gecontacteerde makers geen afspraak maken, worden deze persoonlijk uitgenodigd (via telefoon) ofwel in de pop-up, ofwel in de door de maker gekozen plaats (atelier, kantoor, thuis).

In deze fase werken we ook verder aan onze binding met leveranciers (bestaande 3D-print- en lasercuttingcentra in en buiten de regio, alternatieve samenwerkingen met bv. FabLabs …). Hier hebben we aandacht voor de mogelijkheden en beperkingen van deze leveranciers: prijzen, kwaliteit, machinepark en technologieën …).
% subsubsection teasers_voor_makers_erkend_door_handmade_in_brugge (end)

\subsubsection{Events, samenwerking met `Handmade in Brugge', stad Brugge…} % (fold)
\label{ssub:events_samenwerking_met_handmade_in_brugge_stad_brugge}
Mirror bouwt aan de samenwerking en bekendheid van `Handmade in Brugge', de Stad Brugge en mogelijk andere partners.

In de pop-up zet Mirror zich in de schaduw van de ontwerpers. De Brugse makers zoeken zelf nog naar naambekendheid en uitbreiding van hun markt. Door ons hier bewust in de schaduw te stellen, openen we deuren naar de makers die zelf graag in de spotlight staan. Mirror organiseert snel wisselende thematische tentoonstellingen van makers, organiseert ‘makers-avonden’ waar het grote publiek kan kennismaken en luisteren naar gedreven ontwerpers…

Hier heeft Mirror aandacht voor de door `Handmade in Brugge' niet erkende makers. Zij krijgen persoonlijke uitnodigingen voor de events en kunnen intekenen voor een eigen gepersonaliseerde teaser via een wedstrijd.

Mirror investeert in deze fase in persoonlijke contacten en gesprekken met makers in en buiten de pop-up. Om de makers die zelf een onderneming hebben te kunnen ontmoeten heeft de pop-up afwijkende openingsuren (13u30-18u en 18u30-22u).

In de schaduw van deze evenementen bouwen we aan onze eigen naambekendheid bij de makers en breiden we zo ons netwerk uit. We tasten en voelen naar noden bij makers en verfijnen zo onze dienstverlening en product.
% subsubsection events_samenwerking_met_handmade_in_brugge_stad_brugge (end)

\subsubsection{Website} % (fold)
\label{ssub:website}
De mogelijkheden die Mirror biedt worden grafisch vormgegeven in een uitgepuurde website.

De teasers voor de ontwerpers worden op de site geplaatst en lichten op wanneer afgehaald. Zo worden de ontwerpers/makers in het daglicht geplaatst. Een win-win voor beide partijen.

De georganiseerde events en partnerorganisaties worden uitgebreid toegelicht.

De website speelt ook een grote rol op het social medialandschap waar er zowel met guerrillamarketing als met traditionele reclame aandacht vestigen aan de makers in Brugge, en door unieke Mirror-geproduceerde producten aan te kunnen bieden en tonen aan klanten zal er aan beide kanten een interesse voor de dienst van Mirror komen.
% subsubsection website (end)

\subsubsection{Uitbreiding targetgroepen} % (fold)
\label{ssub:uitbreiding_targetgroepen}
Geleidelijk aan worden de verwante doelgroepen meegenomen in het groeipad van Mirror. Zo worden docenten en studenten van onderwijs-en opleidingscentra gericht uitgenodigd op events in de pop-up.

De architecten zijn een nevenverhaal. Deze targetgroep benaderen we apart en nog niet in deze fase van ons ondernemingsschap.
% subsubsection uitbreiding_targetgroepen (end)

% subsection startfase_pop_up (end)

\subsection{Mirror als volwaardige zaak} % (fold)
\label{sub:mirror_als_volwaardige_zaak}

\subsubsection{Brugge} % (fold)
\label{ssub:brugge}
Wanneer de periode van de pop-up naar zijn einde gaat en Mirror een verfijnd zicht heeft op de noden en behoeftes van de makers start de volgende fase. In deze fase is de `winkel'-functie overbodig en werkt Mirror vanuit een kantoor. De locatie van dit kantoor is afhankelijk van verschillende factoren: bereikbaarheid en klantnabijheid, trending locatie, prijs… Er dient ook rekening gehouden te worden met de groei van Mirror, wanneer we naast de dienstverlening de productie stapsgewijs uitbouwen.
% subsubsection brugge (end)

\subsection{Uitbreiding concept} % (fold)
\label{sub:uitbreiding_concept}
Wanneer Mirror `werkt' in Brugge kan gedacht worden aan uitbreiding naar andere Vlaamse en buitenlandse  steden. Dit kan in bijhuizen of franchising. Deze opties worden in deze fase nog niet verder onderzocht.
% subsection uitbreiding_concept (end)

% subsection mirror_als_volwaardige_zaak (end)

% section actieplan (end)
